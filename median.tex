\documentclass[
	11pt,
	DIV10,
	ngerman,
	a4paper,
	oneside,
	headings=normal,
	captions=tableheading,
	final,
	numbers=noenddot
]{scrartcl}


\usepackage[ruled]{algorithm2e}
\usepackage{graphicx}
\usepackage{hyperref}
\usepackage{amsmath}
\usepackage{mathtools}
\DeclarePairedDelimiter{\ceil}{\lceil}{\rceil}
\DeclarePairedDelimiter{\norm}{\lVert}{\rVert}
% \DeclarePairedDelimiter{\brace}{\lbrace}{\rbrace}

\title{A Branch-and-Price Solver based on Column Generation}
\subtitle{\vspace{0.5cm}a Solution to the Capacitated P-Median Problem}
\author{Yiran Xing, Yinglun Liu}


\begin{document}
\maketitle


\section{Problem and Formulation}
\label{sec1}

In the classic capacitated p-median problem, we are given a number $ n \in N $ of locations and a number $ p \in N $ of clusters. For each location $ i $, real numbers $ q_{i} \geq 0 $ and $ Q_{i} \geq 0 $ are given to indicate the demand and capacity of this particular location. Between each pair $ \left(i, j\right) $ of locations $ i $ and $ j $, a real number $ d_{ij} \geq 0 $ is given to indicate the distance between them. The objective to find a way to

\begin{itemize}
	\item choose $ p $ locations as cluster medians,
	\item assign each location to a single cluster median,
\end{itemize}

such that

\begin{itemize}
    \item the sum of distances $ d_{ij} $ over all pairs $ \left(i, j\right) $ of location $ i $ and cluster median $ j $, such that $ i $ is allocated to $ j $, is minimized,
    \item for each cluster median $ j $, the sum of demands $ q_{i} $ over those locations $ i $ that are allocated to $ j $ does not exceed its capacity $ Q_{j} $.
\end{itemize}

\subsection{Original Problem Formulation}

Let us denote by $ y_{j} \in \{0, 1\} $ whether location $ j $ is chosen as a cluster median and by $ x_{ij} \in \{0, 1\} $ whether location $ i $ is allocated to cluster median $ j $. Then for the original problem we look to find the optimal solution to:

\begin{equation}
	\label{eq1}
	z_{IP}^{*} = min \sum_{i \in [n]} \sum_{j \in [n]} d_{ij} \cdot x_{ij}
\end{equation}

s.t.

\begin{align}
	\label{eq2}	\sum_{j \in [n]} x_{ij} &= 1 \quad \forall i \in [n] \\[1em]
	\label{eq3} \sum_{j \in [n]} y_{j} &= p \\[1em]
	\label{eq4} \sum_{i \in [n]} q_{i} \cdot x_{ij} &\leq Q_{j} \cdot y_{j} \quad \forall j \in [n] \\[1em]
	x_{ij} &\in \{0, 1\} \quad \forall i \in [n], \forall j \in [n] \nonumber \\[1em]
	y_{j} &\in \{0, 1\} \quad \forall j \in [n] \nonumber
\end{align}

Eq. \eqref{eq2} refer to the assignment constraint, i.e. each location is assigned to exactly one cluster. Eq. \eqref{eq3} refer to the $ p $-median constraint that exactly $ p $ locations are selected as cluster medians. Eq. \eqref{eq4} stipulates that location-median assignments always respect the medians' capacities.

\subsection{Master Problem Formulation}

Let us denote by $ \Omega_{j} $ the polyhedron defined by the $ j $th constraint from Eq. \eqref{eq4}:

\small

\begin{equation}
	\Omega_{j} = \left\{(x_{1j}, x_{2j}, ..., x_{nj}, y_{j}) \in R^{n + 1} \hspace{0.5em} | \hspace{0.5em} 0 \leq x_{ij} \leq 1 \hspace{0.5em} \forall i \in [n], \hspace{0.5em} 0 \leq y_{j} \leq 1, \hspace{0.5em} \sum_{i \in [n]} q_{i} \cdot x_{ij} \leq Q_{j} \cdot y_{j}\right\} \nonumber
\end{equation}

\normalsize

Let us denote by $ X_{j} $ the set of integer points within $ \Omega_{j} $:

\begin{equation}
	X_{j} = \Omega_{j} \cap Z^{n + 1} \nonumber
\end{equation}

As in \cite{ceselli2005branch}, every point $ (x_{1j}, x_{2j}, ..., x_{nj}, y_{j}) \in conv(X_{j}) $ can be obtained as a convex combination of the $ L_{j} + 1 $ extreme points of $ conv(X_{j}) $. Let us denote by

\begin{equation}
	\left\{(0, 0, ..., 0, 0), (\bar{x}_{1j}^{1}, \bar{x}_{2j}^{1}, ..., \bar{x}_{nj}^{1}, 1), (\bar{x}_{1j}^{2}, \bar{x}_{2j}^{2}, ..., \bar{x}_{nj}^{2}, 1), ..., (\bar{x}_{1j}^{L_{j}}, \bar{x}_{2j}^{L_{j}}, ..., \bar{x}_{nj}^{L_{j}}, 1)\right\} \nonumber
\end{equation}

the set of extreme points of $ conv(X_{j}) $ and introduce real multipliers $ \lambda_{j}^{0} $, $ \lambda_{j}^{1} $, $ \lambda_{j}^{2} $, ..., $ \lambda_{j}^{L_{j}} $ for each corresponding extreme point. Then we have

\begin{equation}
	(x_{1j}, x_{2j}, ..., x_{nj}, y_{j}) = \sum_{k = 0}^{L_{j}} (\bar{x}_{1j}^{k}, \bar{x}_{2j}^{k}, ..., \bar{x}_{nj}^{k}, y_{j}^{k}) \cdot \lambda_{j}^{k},
\end{equation}

where

\begin{equation}
	\label{eq5}
	\sum_{k = 0}^{L_{j}} \lambda_{j}^{k} = 1, \quad 0 \leq \lambda_{j}^{k} \leq 1 \hspace{0.5em} \forall 0 \leq k \leq L_{j}.
\end{equation}

By substituting the original variables in the original formulation with the expression in Eq. \eqref{eq5} we obtain the Danzig-Wolfe reformulation:

\begin{equation}
	\label{eq6}
	z_{MP}^{*} = min \sum_{j \in [n]} \sum_{k = 0}^{L_{j}} \lambda_{j}^{k} \cdot \sum_{i \in [n]} d_{ij} \cdot x_{ij}^{k}
\end{equation}

s.t.


\subsection{Pricing Problem Formulation}



\section{Experiments}
\label{sec2}

\bibliographystyle{alpha}
\bibliography{references}

\end{document}
