\documentclass[
	11pt,
	DIV10,
	ngerman,
	a4paper,
	oneside,
	headings=normal,
	captions=tableheading,
	final,
	numbers=noenddot
]{scrartcl}


\usepackage[ruled]{algorithm2e}
\usepackage{graphicx}
\usepackage{hyperref}
\usepackage{amsmath}
\usepackage{mathtools}
\DeclarePairedDelimiter{\ceil}{\lceil}{\rceil}
\DeclarePairedDelimiter{\norm}{\lVert}{\rVert}

\title{A Branch-and-Price Solver based on Column Generation}
\subtitle{\vspace{0.5cm}a Solution to the Capacitated P-Median Problem}
\author{Yiran Xing, Yinglun Liu}


\begin{document}
\maketitle


\section{Problem and Formulation}
\label{sec1}

In the classic capacitated p-median problem, we are given a number $ n \in N $ of locations and a number $ p \in N $ of clusters. For each location $ i $, real numbers $ q_{i} \geq 0 $ and $ Q_{i} \geq 0 $ are given to indicate the demand and capacity of this particular location. Between each pair $ \left(i, j\right) $ of locations $ i $ and $ j $, a real number $ d_{ij} \geq 0 $ is given to indicate the distance between them. The objective to find a way to

\begin{itemize}
	\item choose $ p $ locations as cluster medians,
	\item assign each location to a single cluster median,
\end{itemize}

such that

\begin{itemize}
    \item the sum of distances $ d_{ij} $ over all pairs $ \left(i, j\right) $ of location $ i $ and cluster median $ j $, such that $ i $ is allocated to $ j $, is minimized,
    \item for each cluster median $ j $, the sum of demands $ q_{i} $ over those locations $ i $ that are allocated to $ j $ does not exceed its capacity $ Q_{j} $.
\end{itemize}

\subsection{Original Problem Formulation}
\subsection{Master Problem Formulation}


\cite{ceselli2005branch}

\section{Experiments}
\label{sec2}

\bibliographystyle{alpha}
\bibliography{references}

\end{document}
